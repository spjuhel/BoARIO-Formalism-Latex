% Pour la table des matières
\usepackage[nohints,tight]{minitoc}		% Mini table des matières, en français
\setcounter{minitocdepth}{2} % Mini-toc détaillées (sections/sous-sections)
\setlength{\mtcindent}{-1em} % décalage des minitoc à gauche
\dominitoc{}

\usepackage[nottoc]{tocbibind} % pour que la bibliographie apparaisse dans la table des matières (avec l'option pour que la table des matières elle-même n'apparaisse pas dans la table des matières).
% \usepackage{tocloft}% pour pouvoir modifier les tailles d'espacement dans la table des matières
\usepackage[titles]{tocloft}
\setlength{\cftchapnumwidth}{6.5em} % Adjust the value as needed
\renewcommand{\cftchappresnum}{Chapter\ }
\renewcommand{\cftchapaftersnum}{\ --}
% Fix space between numbering and figure caption in list of figures %%%
\setlength{\cftfignumwidth}{3em}
\setlength{\cfttabnumwidth}{3em}


%%%%%%%%%%% Header / Foot %%%%%%%%%%%
\usepackage{fancyhdr,emptypage} % garantit que les pages blanches avant les débuts de chapitres soient vraiment blanches (pas d'en-tête ni de pied de page)
\let\cleardoublepage\clearpage

\fancypagestyle{plain}{ %% Page chapitre, toc ...
    \fancyhead{}\fancyfoot[C]{\thepage}
    \renewcommand{\headrulewidth}{0pt}
    \renewcommand{\footrulewidth}{0pt}
}

%%%%%%%%%%% Page normale
\pagestyle{fancy}
    % \renewcommand{\chaptermark}[1]{\markboth{\chaptername \ \thechapter.\ #1}{}} % sert à personnaliser l'affichage de \leftmark (ici : le mot "Chapitre", le numéro, un point, et le titre du chapitre, sans écrire en majuscules)
    % \renewcommand{\chaptermark}[1]{\markleft{\chaptername \ \thechapter.\ #1}{}}
    % \renewcommand{\sectionmark}[1]{\markright{\thesection.\ #1}} % sert à personnaliser l'affichage de \rightmark (ici : le numéro et le titre de la section en cours, sans écrire en majuscules)
    \fancyhf{} % assure que les entête et pieds de page sont vides au départ
    \fancyhead[LE]{\selectfont\nouppercase{\leftmark}}
    \fancyhead[RO]{\selectfont\nouppercase{\rightmark}}
    \fancyfoot[C]{\thepage}
% Explications :
% L = left, R = right, C = center, E = even pages, O = odd pages
%\leftmark : adds name and number of the current top-level structure (for example, Chapter for reports and books classes; Section for articles ) in uppercase letters.
%\rightmark : adds name and number of the current next to top-level structure (Section for reports and books; Subsection for articles) in uppercase letters.

%%%%%%%%%%% Personnaliser les premières pages des chapitres
\usepackage[Lenny]{fncychap}
\ChNameVar{\fontsize{25}{25}\usefont{OT1}{phv}{m}{n}\selectfont}
\ChRuleWidth{0pt}
\ChNumVar{\fontsize{60}{62}\selectfont\textcolor{curcolor}}

\makeatletter
\ChTitleVar{\Huge\rm}
\renewcommand{\DOCH}{%
\setlength{\fboxrule}{\RW} % Let fbox lines be controlled by
\fbox{\CNV\FmN{\@chapapp}\space \CNoV\thechapter}\par\nobreak{}
\vskip 20\p@}
\renewcommand{\DOTIS}[1]{%
\CTV\bfseries\FmTi{#1}\par\nobreak{}
\vskip 20\p@}
\makeatother

\renewcommand{\thesection}{\arabic{section}}

\usepackage[hyperref=true,%
      natbib=true,%
			backref=true,date=year,%
			backend=biber,%
			url=false,doi=false,isbn=false,%
			minbibnames=6, % nb min authors in biblio
			maxbibnames=6, % nb max authors in biblio
			maxcitenames=1,mincitenames=1, % nb min authors as textual
			maxalphanames=1, %nb author ref
			style=authoryear,%authoryear, numeric ,  alphabetic
			sorting=nyt,%
      refsegment=chapter % references per chapters
      ]{biblatex}%

\renewcommand*{\bibfont}{\footnotesize}

%%%%%%%%%%%%%%%%%%%%%%%%%%%%%%%%%%%%%%%%%%%%%%%%%%%%%%%%%%%%%%%%%%%%
%%%%%%%%%%% Bibliography ref  %%%%%%%%%%%
\setlength\bibitemsep{\itemsep}
\renewbibmacro{in:}{} % remove In

\renewcommand*{\labelalphaothers}{}
\DeclareLabelalphaTemplate{
  \labelelement{
    \field[final]{shorthand}
    \field{labelname}
    \field{label}
  }
  \labelelement{\literal{,\addhighpenspace}}
  \labelelement{\field{year}}
}

\AtEveryBibitem{%
    \clearfield{note} % Remove note
    \clearlist{language} % Remove doi
}

% >> bib field filter
\DeclareSourcemap{
    \maps[datatype=bibtex]{
        % remove fields that are always useless
        \map{
            \step[fieldset=abstract, null]
            \step[fieldset=pagetotal, null]
            \step[fieldset=note, null]
        }
        % remove URLs for types that are primarily printed
        \map{
            \pernottype{software}
            \pernottype{online}
            \pernottype{report}
            \pernottype{techreport}
            \pernottype{standard}
            \pernottype{manual}
            \pernottype{misc}
            \step[fieldset=url, null]
            \step[fieldset=urldate, null]
        }
        \map{
            \pertype{inproceedings}
            % remove mostly redundant conference information
            \step[fieldset=venue, null]
            \step[fieldset=eventdate, null]
            \step[fieldset=eventtitle, null]
            % do not show ISBN for proceedings
            \step[fieldset=isbn, null]
            % Citavi bug
            \step[fieldset=volume, null]
        }
      }
    }
% <<

%%%%%%%%%%
%%% Local Variables:
%%% mode: LaTeX
%%% TeX-master: "../main"
%%% End:
